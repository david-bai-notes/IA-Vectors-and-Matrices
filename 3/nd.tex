\section{Vectors in Euclidean Spaces of High Dimensions}
\subsection{Vectors in the Real Vector spaces}
If we regard $\mathbb R^n$ algebraically, i.e. see vectors as just a set of components, then it is easy to generalize from $3$ to $n$ dimensions.
\begin{definition}
    Let $\mathbb R^n$ be the set of real $n$-tuples.
    We define addition by $(x_1,x_2,\ldots,x_n)+(y_1,y_2,\ldots,y_n)=(x_1+y_1,x_2+y_2,\ldots, x_n+y_n)$ and scalar multiplication by $\lambda(x_1,x_2,\ldots,x_n)=(\lambda x_1,\lambda x_2,\ldots,\lambda x_n)$.\\
    So we can define linear combinations and the notion of parallel similarly.
\end{definition}
For any $\underline{x}\in\mathbb R^n$, we can write $\underline{x}=\sum_ix_i\underline{e_i}$ where $\underline{e_i}$ is a set of orthorgonal basis.
For example, we can take $e_1=(1,0,\ldots,0), e_2=(0,1,\ldots,0), \ldots, e_n=(0,0,\ldots,1)$.
This is called the standard basis of $\mathbb R^n$.\\
We can define the inner (dot) product in a similar way:
\begin{definition}
    The inner product (aka scalar product, dot product) is defined by
    $$\underline{x}\cdot\underline{y}=\sum_ix_iy_i$$
\end{definition}
We have a few properties for the inner products, which is basically analogous to the case in the $3$-dimensional case.
\begin{proposition}
    1. The inner product is symmetric.\\
    2. The inner product is bilinear.\\
    3. $\underline{x}\cdot\underline{x}\ge 0$ and the equality hold if and only if $\underline{x}=\underline{0}$.
    That is, it is positive definite.
    We thus define the length or norm $|\underline{x}|$ to be $\sqrt{\underline{x}\cdot\underline{x}}$.
    4. The standard basis in $\mathbb R^n$ is an orthonormal basis.
\end{proposition}
\begin{proof}
    Trivial.
\end{proof}
\begin{theorem}[Cauchy-Schwartz Inequality]
    $$|\underline{x}\cdot\underline{y}|\le |\underline{x}||\underline{y}|$$
    The equality hold if and only if $\underline{x},\underline{y}$ are parallel to each other.
\end{theorem}
The deduction from this is that we can define the angle between two vectors by examining the ratio between the right and left hand side of the inequality.\\
We can also have the triangle inequality, as one may expect norms to satisfy.
\begin{proof}
    Consider
    $$0\le|\underline{x}-\lambda\underline{y}|^2=|x|^2-2\lambda\underline{x}\cdot\underline{y}+\lambda^2|y|^2$$
    Taking this as a quadratic in $\lambda$, we have
    $$\delta=4(x\cdot y)^2-4|x|^2|y|^2\le 0\implies|\underline{x}\cdot\underline{y}|\le |\underline{x}||\underline{y}|$$
    The equality holds if and only if $\delta=0\iff 0=|\underline{x}-\lambda\underline{y}|^2\iff\underline{x}\parallel\underline{y}$.
\end{proof}
\begin{theorem}
    $|\underline{x}+\underline{y}|\le|\underline{x}|+|\underline{y}|$
\end{theorem}
\begin{proof}
    Square both sides and use Cauchy-Schwartz.
\end{proof}
Note that by $\underline{x}\cdot\underline{y}$, we can think of $\underline{x},\underline{y}$ as both row and column vectors, since it doesn't matter.
But still if we take them as for example column vectors, then their transposes are row vectors, so $\underline{x}\cdot\underline{y}=\underline{x}^\top\underline{y}$.\\
The algebraic definition of of the inner product can be also written by $\delta_{ij}$, so it gives us the initiative to generalize the summation convention.
In $\mathbb R^3$, we also have an algebraic definition of cross product. but we cannot really generalize it to $\mathbb R^n$.
We have a generalization of $\epsilon$ though, which is also antisymmetric.\\
But in $\mathbb R^2$, this generalization gives $\epsilon_{ij}$, where we can define a product by
$$[\underline{a},\underline{b}]=\epsilon_{ij}a_ib_j=a_1b_2-a_2b_1$$
Geometrically, this gives the signed area of the parallelogram that $a,b$ defined.\\
In comparison, $[\underline{a},\underline{b},\underline{c}]=\underline{a}\cdot(\underline{b}\times\underline{c})=\epsilon_{ijk}a_ib_jc_k$ is the volume of the size of a parallelopiped that these three vectors construct.
\subsection{Axioms of Real Vector Spaces}
\begin{definition}
    Let $V$ be a set of objects called vectors with operations:\\
    1. $\underline{v}+\underline{w}\in V$.\\
    2. $\lambda\underline{v}\in V$.\\
    For $\underline{v},\underline{w}\in V, \lambda\in\mathbb R$.\\
    Then $V$ is called a real vector space if $(V,+,\underline{0})$ is an abelian group addition\\
    1. $\lambda (\underline{v}+\underline{w})=\lambda\underline{v}+\lambda\underline{w}$.\\
    2. $(\lambda+\mu)\underline{v}=\lambda\underline{v}+\mu\underline{v}$.\\
    3. $(\lambda\mu)\underline{v}=\lambda(\mu(\underline{v}))$.\\
    4. $1\underline{v}=\underline{v}$.\\
    where $\lambda,\mu\in\mathbb R, \underline{v},\underline{w}\in V$
\end{definition}
\begin{definition}
    For any vectors $\underline{v_1},\underline{v_2},\ldots\underline{v_r}\in V$, we can form a linear combination
    $$\sum_{i=1}^ra_i\underline{v_i}$$
    where $a_i\in\mathbb R$.
    The span of these vectors $\operatorname{span}\{v_1,v_2,\ldots,v_r\}$ consists of all linear combinations of these vectors.\\
    The span is a subspace, that is, a subset of the vector space that is itself a vector space under the same way of vector addition and scalar multiplication.
\end{definition}
It is immediate that nonempty subset $U\subseteq V$ is a subspace if and only if $\operatorname{span} U=U$.
\begin{example}
    Take $V=\mathbb R^3$, then any line or plane through the origin is a subspace, but a line or plane that does not contain the origin is not since \underline{0}=$\underline{v}+(-1)\underline{v}\in\operatorname{span}\{\underline{v}\}$.
\end{example}
\begin{definition}
    A set of vectors $\underline{v_1},\underline{v_2},\ldots,\underline{v_r}$, a linear relation of them is an equation
    $$\sum_{i=1}^r\lambda_i\underline{v_i}=\underline{0}$$
    If the equation if true only if $\lambda_i=0$ for every $i$, then the vectors are called linearly independent and that they obey only the trivial linear relation.
    And we say this set of vectors is a independent set.\\
    Otherwise, we say they are linearly dependent, and the set of vectors a dependent set.
\end{definition}
\begin{example}
    1. So for example, if we take $V=\mathbb R^2$ and consider the set $(0,1),(1,0),(0,2)$.
    It is a dependent set since $2(0,1)-(0,2)=(0,0)$.\\
    2. Any set containing $\underline{0}$ is dependent.\\
    3. $\{\underline{a}\}$ is independent if and only if $\underline{a}\neq\underline{0}$.\\
    4. $\{\underline{a},\underline{b}\}$ is independent if and only if $\underline{a}\nparallel\underline{b}$.
\end{example}
\begin{definition}
    A function $\cdot:V\times V\to\mathbb R$ is called an inner product on $V$ if and only if:\\
    1. $\underline{v}\cdot\underline{w}=\underline{w}\cdot\underline{v}$.\\
    2. It is bilinear.\\
    3. $\underline{v}\cdot\underline{v}\ge 0$ and the equality hold if and only if $\underline{v}=\underline{0}$.
\end{definition}
\subsection{Basis and Dimension}
For general vector spaces, a basis $\mathscr{B}$ is a independent set of vectors such that $\operatorname{span}\mathscr{B}=V$.
Given this property, it is trivial that the coefficients of any vector as a linear combination of elements in $\mathscr{B}$ are unique.
\begin{example}
    The standard basis for $\mathbb R^n$ consisting of
    $$\underline{e_1}=(1,0,\ldots,0),\underline{e_2}=(0,1,\ldots,0),\ldots,\underline{e_n}=(0,0,\ldots,1)$$
    is a basis.
    There are many other basis can be chosen though.
    For example, in $\mathbb R^2$, $\{(1,0),(1,1)\}$, $\{(1,-1),(1,1)\}$ or simply any $\{\underline{a},\underline{b}\}$ that are not parallel would be a basis.
\end{example}
\begin{theorem}
    If $\{\underline{e_1},\underline{e_2},\ldots.\underline{e_n}\}$ and $\{\underline{f_1},\underline{f_2},\ldots.\underline{f_m}\}$, then $m=n$.
\end{theorem}
It follows that we can define the following
\begin{definition}
    The number of vectors in a basis in a vector space is called its dimension.
    So with this definition, $\mathbb R^n$ has dimension $n$.
\end{definition}
Now we can prove the theorem.
\begin{proof}
    Note that we can find coefficients $A_{ai}$ such that $\underline{f_a}=\sum_iA_{ai}\underline{e_i}$ for each $a$.
    Similarly, $\underline{e_i}=\sum_aB_{ia}\underline{f_a}$.\\
    Then $\underline{f_a}=\sum_b(\sum_iA_{ai}B_{ib})\underline{f_b}$ and $\underline{e_i}=\sum_j(\sum_aB_{ia}A_{aj})\underline{e_j}$.
    So $\sum_iA_{ai}B_{ib}=\delta_{ij}$ and $\sum_aB_{ia}A_{aj}=\delta_{ij}$.\\
    $\sum_{i,a}A_{ai}B_{ia}=\sum_i\delta_{ii}$ and $\sum_{i,a}A_{ai}B_{ia}=\sum_a\delta_{aa}$, thus $n=m$.
\end{proof}
In fact we did secretly used traces, but we have presented it within the scope of this course.
The proof in the general case can be done elegantly using matrices.\\
We can also apply our notions to the following:
\begin{proposition}
    Let $V$ be a vector space of dimension $n$.\\
    1. If $Y=\{\underline{w_1},\underline{w_2},\ldots,\underline{w_m}\}$ spans $V$, then $m\ge n$ and if $m>n$ we can remove an element from $Y$ such that the new set of vectors still spans $V$.
    Thus we can continue doing it till it becomes a basis.\\
    2. If $X=\{\underline{u_1},\underline{u_2},\ldots,\underline{u_k}\}$ be an independent set of vectors, then $k\le n$ and if $k<n$ we can add vector to $X$ until we have a basis.
\end{proposition}
\begin{proof}
    1. If $Y$ is linearly independent, then we are done and $m=n$.
    Otherwise, there is some linear relation $\sum_i\lambda_i\underline{w_i}=0$ such that $\exists i, \lambda_i\neq 0$.
    WLOG $i=m$, then $\underline{w_m}=\lambda_m^{-1}(\sum_{i,i\neq m}\underline{w_i})$.
    So $Y'=Y\setminus\{\underline{w_m}\}$ spans $V$.
    We can repeat this till it becomes independent, i.e. we obtain a basis.\\
    2. If $X$ spans $V$, we are done and then $k=m$.
    Otherwise, there is some $\underline{u_{k+1}}\in V\setminus\operatorname{span}X$, then $X\cup\{\underline{u_{k+1}}\}$ is still independent.
    Indeed, if there is some nontrivial linear relation $\sum_i\mu_i\underline{u_i}=0$, then $\mu_{k+1}\neq 0$ since $X$ is independent, but then $\underline{u_{k+1}}$ can be written as a linear combination of $\underline{u_i}$ for $i\in\{1,2,\ldots,k\}$, contradiction.\\
    So we can do it over again and obtain a basis at last.
\end{proof}
Note that the basis does not use nor depend on the inner product structure of the vector space.
But we can make use of it to obtain an orthorgonal basis.
\begin{definition}
    A basis is called orthorgonal if they are pairwisely orthorgonal.
\end{definition}
\begin{proposition}
    Any set of pairwise orthogonal vectors is independent.
\end{proposition}
\begin{proof}
    Easy.
\end{proof}
\subsection{Vectors in the Complex Vector Space}
\begin{definition}
    Let $\mathbb C^n$ consist of all complex $n$-tuples.
    We define the vector space by taking the vector addition and scalar multiplication analogously to the $\mathbb R^n$ case.
\end{definition}
Taking real scalars in scalar multiplication, then $\mathbb C^n$ is just a real vector space of dimension $2n$.
Taking complex scalars, $\mathbb C^n$ is a complex vector space.\\
The definitions of linear combinations, linear independence, basis, dimensions are all analogous.
Consider
$$(z_1,\ldots,z_n)=(x_1+iy_1,\ldots,x_n+iy_n)=(x_1,\ldots,x_n)+i(y_1,\ldots,y_n)$$
then $\underline{e_j}$ and $i\underline{e_j}$ gives a basis for $\mathbb C^n$ as a real vector space.
And the $\underline{e_j}$ is a basis for $\mathbb C^n$ as a complex vector space.
We view $\mathbb C_n$ to be over $\mathbb C$ unless otherwise stated.
\begin{definition}
    The inner product on $\mathbb C^n$ is defined by
    $(\underline{z},\underline{w})=\sum_{j}\bar{z}_jw_j$
\end{definition}
\begin{proposition}
    The complex inner product is:\\
    1. Hermitian, $(\underline{z},\underline{w})=\overline{(\underline{w},\underline{z})}$.\\
    2. Anti-linear, $(\underline{z},\lambda\underline{w}+\lambda'\underline{w'})=\lambda(\underline{z},\underline{w})+\lambda'(\underline{z},\underline{w'})$ and $(\lambda\underline{z}+\lambda'\underline{z'},\underline{w})=\bar\lambda(\underline{z},\underline{w})+\bar\lambda'(\underline{z'}+\underline{w})$.\\
    3. Positive definite.
\end{proposition}
The geometric content of this sort of inner product is a little more subtle than in the real case.
\begin{example}
    Consider the complex inner product on $\mathbb C$, that is $n=1$, since there is only one component, $(z,w)=\bar zw$.
    Suppose $z=a_1+ia_2,w=b_1+ib_2$ where $a_1,a_2,b_1,b_2\in\mathbb R$.
    Let $\underline{a}=(a_1,a_2),\underline{b}=(b_1,b_2)$, then
    $$(z,w)=a_1b_1+a_2b_2+i(a_1b_2-a_2b_1)=\underline{a}\cdot\underline{b}+i[\underline{a},\underline{b}]$$
\end{example}
Given the positive definite property, we can define the length or norm $|\underline{z}|$ for any $\underline{z}\in\mathbb C^n$ by $|\underline{z}|=\sqrt{(\underline{z},\underline{z})}$.
We still say that two complex vectors are orthorgonal if their (complex) inner product vanishes.
In this way, the standard basis for $\mathbb C^n$ is orthorgonal.\\
If $\underline{z_1},\underline{z_2},\ldots,\underline{z_k}$ are nonzero and orthorgonal, then they are linearly independent.\\
Notationally, we think of the vectors in $\mathbb C^n$ as column vector, and the Hermitian conjugate $\underline{z}^{\dagger}$ the row vector consisting of the conjugates of the entries.