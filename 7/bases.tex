\section{Change of Bases, Canonical Forms and Symmetries}
\subsection{Change of Bases in General}
Recall that given a linear operator $T:V\to W$, we can write its matrix form once we have specified the bases $\{\underline{e_1},\underline{e_2},\ldots,\underline{e_n}\}$ and $\{\underline{f_1},\underline{f_2},\ldots,\underline{f_m}\}$ for $V,W$.
So $T(\underline{e_i})=A_{ai}\underline{f_a}$ for some matrix $A$,
For other bases $\{\underline{e_i'}\}$ and $\{\underline{f_i'}\}$ for $V,W$, we have a matrix $A'$ with $T(\underline{e_i'})=A'_{ai}\underline{f_a'}$.
Suppose the bases are related by $\underline{e_i'}=P_{ji}\underline{e_j}, \underline{f_a'}=Q_{ba}\underline{f_b}$.
So $P,Q$ are invertible and
\begin{proposition}
    $A'=Q^{-1}AP$.
\end{proposition}
We say $P,Q$ constitutes a change of bases.
The column $i$ of $A$ consists of components of $T(\underline{e_i})$ with respect to the basis $\{\underline{f_a}\}$, column $i$ of $P$ consists of components of new basis vectors with respect to the old basis, similar for $Q$.
\begin{example}
    Let $\dim V=n=2$, $\dim W=m=3$, and consider the linear map $T$ with $T(\underline{e_1})=\underline{f_1}+2\underline{f_2}-\underline{f_3}$ and $T(\underline{e_2})=-\underline{f_1}+2\underline{f_2}+\underline{f_3}$, so $T$ has matrix
    $$A=\begin{pmatrix}
        1&-1\\
        2&2\\
        -1&1
    \end{pmatrix}$$
    Under these choice of bases.
    Now consider the new basis $\underline{e_i'}$ for $V$ by
    $$\underline{e_1'}=\underline{e_1}-\underline{e_2},\underline{e_2'}=\underline{e_1}+\underline{e_2}$$
    which yields a changes of basis matrix
    $$P=\begin{pmatrix}
        1&1\\
        -1&1
    \end{pmatrix}$$
    And a new basis $\underline{f_i'}$ for $W$ is defined by $$\underline{f_1'}=\underline{f_1}-\underline{f_3},\underline{f_2'}=\underline{f_2},\underline{f_3'}=\underline{f_1}+\underline{f_3}$$
    which has a matrix
    $$Q=\begin{pmatrix}
        1&0&1\\
        0&1&0\\
        -1&0&1
    \end{pmatrix}$$
    So the change of basis formula gives
    $$A'=Q^{-1}AP=\begin{pmatrix}
        2&0\\
        0&4\\
        0&0
    \end{pmatrix}$$
    which can be verified by direct calculation.
\end{example}
\begin{proof}
    For vectors $\underline{y},\underline{x}$ with $\underline{y}=A\underline{x}$ for some invertible matrix (representing a basis) $A$, we denote the components in the respective components by
    $$X=\begin{pmatrix}
        x_1\\
        x_2\\
        \vdots\\
        x_n
    \end{pmatrix},Y=\begin{pmatrix}
        y_1\\
        y_2\\
        \vdots\\
        y_n
    \end{pmatrix}$$
    to avoid confusion.
    So for a change of basis by matrix $P,Q$, the formula $Y=AX$ turns to $Y'=A'X'$, hence, we have $\underline{x}=x_i\underline{e_i}=x_j'\underline{e_j'}=x_j'(\underline{e_i}P_{ij})=P_{ij}x_j'\underline{e_i}$.
    Therefore $X=PX'$, similarly $Y=QY'$, so $A'X'=Y'=Q^{-1}Y=Q^{-1}AX=(Q^{-1}AP)X'$, so $A'=Q^{-1}AP$
\end{proof}
In the special cases where $V=W$ and the bases are the same for $V,W$ before and after the change of basis, then $A$ is transformed to a matrix similar to $A$ by that change-of-basis matrix.
Note that this can justify immediately that $\operatorname{tr}A'=\operatorname{tr}A,\det A'=\det A,\chi_{A'}=\chi_A$ for similar matrices $A,A'$.\\
If $V=W=\mathbb F^n$ where $\mathbb F=\mathbb R$ or $\mathbb C$.
For a diagonalizable matrix $M$, then by changing the standard basis into the eigenbasis, $M$ becomes diagonal.
\begin{proof}[Alternative proof]
    Take the linear map $T$, then
    $$\underline{f_a}Q_{ab}A'_{bi}=\underline{f_b'}A'_{bi}=T(\underline{e_i'})=T(\underline{e_j}P_{ji})=P_{ji}T(\underline{e_j})=\underline{f_a}A_{aj}P_{ji}$$
    So $AP=QA'$ by considering the coefficient of $\underline{f_a}$.
\end{proof}
\subsection{Jordan Canonical/Normal Form}
This result classifies complex $n\times n$ matrices up to similarity (i.e. conjugacy classes).
\begin{proposition}
    Any $2\times 2$ complex matrix $A$ is similar to one of the followings:
    $$
    \begin{pmatrix}
        \lambda_1&0\\
        0&\lambda_2
    \end{pmatrix},\lambda_1\neq\lambda_2;
    \begin{pmatrix}
        \lambda&0\\
        0&\lambda
    \end{pmatrix};
    \begin{pmatrix}
        \lambda&1\\
        0&\lambda
    \end{pmatrix}
    $$
    Where $\lambda_1,\lambda_2,\lambda\in\mathbb C$.
\end{proposition}
\begin{proof}
    If $A$ is diagonalizable (which includes the case where $\chi_A$ has two distinct roots) then the proposition is immediate.
    Otherwise, $\chi_A(t)=c(t-\lambda)^2,c\neq 0$ and $\operatorname{null}(A-\lambda I)=1$ and we shall show that $A$ is similar to a matrix of the third form.
    Indeed, let $\underline{v}$ be an eigenvector for $\lambda$ and $\underline{w}$ any vector that is independent from $\underline{v}$.
    Then $A\underline{v}=\lambda\underline{v},A\underline{w}=\alpha\underline{v}+\beta\underline{w}$, then the matrix of this transformation under the basis $\underline{v},\underline{w}$ is
    $$\begin{pmatrix}
        \lambda&\alpha\\
        0&\beta
    \end{pmatrix}$$
    But then $\beta=\lambda$ by looking at the eigenvalue of matrices of this form, and $\alpha\neq 0$ by assumption.
    So we can set $\underline{u}=\alpha\underline{v}$, then under the basis $\underline{u},\underline{w}$, we get the matrix of $A$ to be
    $$\begin{pmatrix}
        \lambda&1\\
        0&\lambda
    \end{pmatrix}$$
    As claimed.
\end{proof}
\begin{proof}[Alternative approach for the third case]
    If $A$ has characteristic polynomial of the form $\chi_A(t)=c(t-\lambda)^2$ with $c\neq 0$ and $A\neq \lambda I$, then there is some vector $\underline{w}$ with
    $$(A-\lambda I)\underline{w}=\underline{u}\neq\underline{0}$$
    but by Cayley-Hamilton, we have
    $$(A-\lambda I)\underline{u}=(A-\lambda I)^2\underline{w}=\underline{0}$$
    so $A$ is of the form under the basis $\underline{u},\underline{w}$.
\end{proof}
\begin{example}
    Consider
    $$A=\begin{pmatrix}
        1&4\\
        -1&5
    \end{pmatrix}$$
    Then $\lambda=3$ and we can have $\underline{w}=(1,0)^\top$, $\underline{u}=(-2,-1)^\top$, therefore
    $$\begin{pmatrix}
        3&1\\
        0&3
    \end{pmatrix}=
    \begin{pmatrix}
        -2&1\\
        -1&0
    \end{pmatrix}^{-1}
    \begin{pmatrix}
        1&4\\
        -1&5
    \end{pmatrix}
    \begin{pmatrix}
        -2&1\\
        -1&0
    \end{pmatrix}$$
\end{example}
\begin{theorem}
    Any $n\times n$ complex matrix is similar to a matrix of the form
    $$A'=
    \begin{pmatrix}
        J_{n_1}(\lambda_1)&&\\
        &\ddots&\\
        &&J_{n_r}(\lambda_r)
    \end{pmatrix}$$
    where $\lambda_1,\ldots,\lambda_r$ are the eigenvalues, $n_1+n_2+\cdots+n_r=n$ and $J_p(\lambda)$, the Jordan Block, is of the form
    $$J_p(\lambda)=\begin{pmatrix}
        \lambda&1&&&\\
        &\lambda&1&&\\
        &&\lambda&&\\
        &&&\ddots&1\\
        &&&&\lambda
    \end{pmatrix}$$
\end{theorem}
\subsection{Quadrics and Conics}
A quadric in $\mathbb R^n$ is a hypersurface defined by
$$Q(\underline{x})=\underline{x}^\top A\underline{x}+\underline{b}^\top\underline{x}+c=0$$
For some real symmetric $n\times n$ matrix $A$, $\underline{b}\in \mathbb R^n,c\in\mathbb R$.
So $Q(\underline{x})=A_{ij}x_ix_j+b_ix_i+c$.\\
Consider classifying solutions up to geometrical equivalence, i.e. no distinction up to isometries.
\begin{theorem}
    Any isometry in $\mathbb R^n$ is a composition of translation and orthogonal transformation.
\end{theorem}
\begin{proof}
    Trivial.
\end{proof}
If $A$ in invertible, we can complete the square.
We set $\underline{y}=\underline{x}+A^{-1}\underline{b}$, so $\mathcal{F}(\underline{y})=\underline{y}^\top A\underline{y}=Q(\underline{x})-k$ for some constant $k$.
So $Q(\underline{x})=0\iff \mathcal F(\underline{y})=k$ for some $k$, but the quadratic form is with respect to a new origin.
In this case, we can diagonalize $\mathcal F$ and change the basis as appropriate and think of them as new coordinate (bases).
The case in $\mathbb R^3$ has been discussed in the previous sections and we see that the quadrics are ellipsoids and hyperboloids (in $2$ directions).\\
If $A$ is singular, that is, if $A$ has one or more $0$ eigenvalue, things change.\\
Back in $\mathbb R^2$, if $A$ is invertible, then by our analysis above, we can write it as a (diagonalized) quadratic form $k=\lambda_1x_1^2+\lambda_2x_2^2$.
For $\lambda_1,\lambda_2>0$, $k>0$ gives an ellipse, $k=0$ gives a point and $k<0$ no solution.
For $\lambda_1\lambda_2<0$, so WLOG $\lambda_1>0>\lambda_2$, then we get a hyperbola for any $k\neq 0$, and a pair of lines for $k=0$.\\
If $\det A=0$, then unless $A$ is zero, we have $\lambda_1\neq 0$ and $\lambda_2=0$, so we diagonaize $A$ in the original formula to get
$$\lambda_1x_1'^2+b_1'x_1'+b_2'x_2'+c=0\iff \lambda_1x_1''^2+b_2'x_2'+c'=0$$
By a shift in $x_1$.
If $b_2'=0$, then we again get a pair of lines for $c'<0$, a single line for $c'=0$ and no solution for $c'>0$.
Otherwise $b_2'\neq 0$, it gives a parabola.\\
Note that all these changes of coordinates is prepresented by an isometry.
They are called conics because these shapes can be obtained by slicing the cone.
The general form for conics can be expressed in terms of $a,b$ semi-major and semi-minor axes or in terms of length unit $\ell$ and eccentricity $e$.\\
For ellipses, we have the general form
$$\frac{x^2}{a^2}+\frac{y^2}{b^2}=1$$
WLOG $b>a$, then $b^2=(1-e^2)a^2$ for some $0\le e<1$.
So we take this $e$ to be the eccentricity.
So the foci are at $x=\pm ae$.\\
And for parabola in the form $y^2=4ax$, the focus is at $x=a$ and $e=1$.\\
For hyperbola, i.e.
$$\frac{x^2}{a^2}-\frac{y^2}{b^2}=1$$
So for $b^2=a^2(e^2-1)$ for $e>1$ the eccentricity.
The foci are $x=\pm ae$.
\subsection{Symmetries and Transformation Groups}
Recall from a previous section that $R$ is orthogonal iff $R^\top R=RR^\top=I$ iff it preserves dot products.
The set of such matrices is a group $\operatorname{O}(n)$.
Also, given the properties above and the multiplictivity of $\det$, $\det R=\pm 1$.
And $\operatorname{SO}(n)\le \operatorname{O}(n)$ is the set of orthogonal matrices with determinant $1$.
While orthogonal matrices preserves lengths, angles and volumns (by alternating forms), special orthogonal matrices also preserves orientation.
Reflections belongs in $\operatorname{O}(n)\setminus\operatorname{SO}(n)$.
For a specific $H\in \operatorname{O}(n)\setminus\operatorname{SO}(n)$, any element in $\operatorname{O(n)}$ is of the form $R$ or $RH$ for some $R\in\operatorname{SO}(n)$.
For example, if $n$ is odd, then we can take $H=-I$
We can regard the transformation $x_i'=R_{ij}x_j$ in two ways:\\
The ``active'' way is to say the rotations transform vectors.
Then $\operatorname{SO}(n)\star\underline{x}$ would be a hypersphere.
The components $x_i'$ are components of new vector.\\
By contrast, the ``passive'' point of view is to think about the basis vectors and the vectors that they are mapped to, so $x_i'$ are the components of the same vector $\underline{x}$ wrt a new (rotated) orthonormal basis.
\subsection{2-Dimensional Minkowski Space and Lorentz Transformations}
Consider a new ``inner product'' in $\mathbb R^2$ given by
$$(\underline{x},\underline{y})=x^\top J\underline{y},J=\begin{pmatrix}
    1&0\\
    0&-1
\end{pmatrix}$$
And also we write $\underline{x}=(x_0,x_1)^\top$ and $\underline{y}=(y_0,y_1)^\top$.
Note that all original properties of inner products hold except positive definiteness.
We can choose basis vectors that are orthonormal, which are the standard basis $\underline{e_0},\underline{e_1}$ with $(\underline{e_0},\underline{e_0})=1=-(\underline{e_1},\underline{e_1}), (\underline{e_0},\underline{e_1})=0$.
\begin{definition}
    This ``new inner product'' is called the Minkowski metric, and $\mathbb R^2$ equipped with it is called Minkowski space.
\end{definition}
Consider
$$M=\begin{pmatrix}
    M_{00}&M_{01}\\
    M_{10}&M_{11}
\end{pmatrix}$$
which preserves Minkowski metric if and only if $(M\underline{x},M\underline{y})=(\underline{x},\underline{y})$ if and only if
$$\underline{x}^\top M^\top JM\underline{y}=\underline{x}^\top J\underline{y}$$
for any $\underline{x},\underline{y}\in\mathbb R^2$, which happens iff $M^\top JM=J$.
By taking determinant, $\det(M)=\pm 1$.
So all such $M$ constitutes a group and has a subgroup with $\det M=1$ and $M_00>0$ is the Lorentz group.\\
To determine the general form of $M$ in Lorentz group, we can find it by requiring $M\underline{e_0},M\underline{e_1}$ to be orthonormal in this generalized sense, so $M$ is in the general form
$$M(\theta)=\begin{pmatrix}
    \cosh\theta&\sinh\theta\\
    \sinh\theta&\cosh\theta
\end{pmatrix}$$
Fixed by $(M\underline{e_0},M\underline{e_0})=1$ and $M_{00}>0$.
Note that $M(\theta)M(\phi)=M(\theta+\phi)$.\\
Curves with $(\underline{x},\underline{x})=k$ with $k$ constant are simply hyperbolic.
Since lengths are preserve, $\underline{x'}=M\underline{x}$ must lie on the same hyperbola.\\
Physically, we want to set
$$M(\theta)=\gamma(v)\begin{pmatrix}
    1&v\\
    v&1
\end{pmatrix}$$
where $v=\tanh\theta$, then $|v|<1$ and $\gamma(v)=(1-v^2)^{-1/2}$
Then (with $x_0=t$ the time dimension and $x_1=x$ the space parameter)
$$\begin{cases}
    t'=\gamma(v)(t+vx)\\
    x'=\gamma(v)(x+vt)
\end{cases}$$
This is the Lorentz Transformation, or boost, relating time and space coordinates for observers moving with relative velocity $v<1$.
The $\gamma$ factor here implies time dilation and length contraction.
Note that $M(\theta_3)=M_{\theta_1}M_{\theta_2}$ and $v_i=\tanh(\theta_i)$ gives $v_3=\frac{v_1+v_2}{1+v_1v_2}$ which is consistent with $v<1$.