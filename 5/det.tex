\section{Determinants and Inverses}
\subsection{Introduction}
Consider a linear map $\mathbb R^n\to\mathbb R^n$ given by the matrix $M$.
We want to define an $n\times n$ matrix $\tilde{M}=\operatorname{adj}M$ and a scalar $\det M$ with
$$M\tilde{M}=(\det M)I$$
Furthermore, $\det M$ is the factor by whcih an area in $\mathbb R^2$ or a volumn in $\mathbb R^3$ is scaled.
If $\det M\neq 0$, then  we will have
$$M^{-1}=\frac{1}{\det M}\tilde{M}$$
For $n=2$, we know that
$$\tilde{M}=
\begin{pmatrix}
    M_{22}&-M_{12}\\
    -M_{21}&M_{11}
\end{pmatrix},
\det M=
\begin{vmatrix}
    M_{11}&M_{12}\\
    M_{21}&M_{22}
\end{vmatrix}
=M_{11}M_{22}-M_{12}M_{21}
$$
works.
Note that $\det M\neq 0$ if and only if $M\underline{e_1},M\underline{e_2}$ are linearly independent if and only if $\operatorname{Im}M=\mathbb R^2$ if and only if $\operatorname{rank}M=2$.\\
For $n=3$, we recall that given any $\underline{a},\underline{b},\underline{c}\in\mathbb R^3$, the scalar $[\underline{a},\underline{b},\underline{c}]$ is the volumn of a parallelopiped spanned by $\underline{a},\underline{b},\underline{c}$.
We can also note that the standard basis vectors obey $[\underline{e_i},\underline{e_j},\underline{e_k}]=\epsilon_{ijk}$.
Note that for $M$ a real $3\times 3$ matrix, its columns are $M\underline{e_i}=M_{ji}\underline{e_j}$.
So the volumn is scaled by a factor
$$[M\underline{e_1},M\underline{e_2},M\underline{e_3}]=M_{i1}M_{j2}M_{k3}[\underline{e_i},\underline{e_j},\underline{e_k}]=M_{i1}M_{j2}M_{k3}\epsilon_{ijk}$$
We define this to be $\det M$.
We can also define $\tilde{M}$ by
$$\underline{R_1}(\tilde{M})=\underline{C_2}(M)\times\underline{C_3}(M)$$
$$\underline{R_2}(\tilde{M})=\underline{C_3}(M)\times\underline{C_1}(M)$$
$$\underline{R_3}(\tilde{M})=\underline{C_1}(M)\times\underline{C_2}(M)$$
So one can see immediately that
$$(\tilde{M}M)_{ij}=\underline{R_i}(\tilde{M})\cdot\underline{C_j}(M)=\det M\delta_{ij}$$
as desired.\\
How about when we consider $n$ in general?
\subsection{Alternating forms}
We first want to generalize our $\epsilon$ symbol to higher dimensions by considering the permutation.
\begin{definition}
    A permutation $\sigma:\{1,2,\ldots,n\}\to\{1,2,\ldots,n\}$ is a bijection of $\{1,2,\ldots,n\}$ to itself.
\end{definition}
So $\{1,2,\ldots,n\}=\{\sigma{1},\sigma{2},\ldots,\sigma{n}\}$.
It is immediate that the permutations on $n$ letters form a group $S_n$ under composition, and it is easy that $|S_n|=n!$.
\begin{definition}
    A permutaion $\tau\in S_n$ is called a transposition of $i,j\in\{1,2,\ldots,n\}$ if $\tau(i)=j,\tau(j)=i,\forall k\neq i,j, \tau(k)=k$.
    We denote $\tau$ by $(p\ q)$.
\end{definition}
\begin{proposition}
    Any permutation is a product of transpositions.
\end{proposition}
\begin{proof}
    Trivial.
\end{proof}
The way we write it is not unique, but the number of transpositions is unique modulo $2$.
\begin{proposition}
    If some permutation $\sigma$ can be written as the product of $k$ transpositions and the product of $l$ transpositions, then $k\equiv l\pmod{2}$.
\end{proposition}
\begin{proof}
    Will see in groups (actually quite trivial).
\end{proof}
\begin{definition}
    We say the permutation $\sigma$ is even if it can be written as the product of an even number of transpositions, odd if otherwise.
    We define the sign, or signature function $\epsilon:S_n\to\{1,-1\}$ by
    $$\epsilon(\sigma)=
    \begin{cases}
        1\text{, if $\sigma$ is even}\\
        -1\text{, otherwise}
    \end{cases}
    $$
\end{definition}
Note that $\epsilon(\operatorname{id})=0$ and more generally $\epsilon(\sigma\circ\pi)=\epsilon(\sigma)\epsilon(\pi)$ for permutations $\sigma,\pi$.
\begin{definition}
    The $\epsilon$ symbol (or tensor) on $n$ letters is defined as
    $$
    \epsilon_{ij\ldots kl}=
    \begin{cases}
        \epsilon(\sigma)\text{, if $ij\ldots kl$ is a permutation $\sigma$ of $\{1,2,\ldots,n\}$}\\
        0\text{, otherwise. That is, some indices coincide.}
    \end{cases}
    $$
\end{definition}
So we now can define the alternating forms.
\begin{definition}
    Given vectors $\underline{v_1},\underline{v_2},\ldots,\underline{v_n}$ in $\mathbb R^n$ or $\mathbb C^n$.
    We define the alternating form to be the scalar
    $$[\underline{v_1},\underline{v_2},\ldots,\underline{v_n}]=\epsilon_{ij\ldots kl}(\underline{v_1})_i(\underline{v_2})_j\cdots(\underline{v_{n-1}})_k(\underline{v_n})_l$$
    One can check that the alternating forms when $n=2,3$ are exactly the same as we have defined them before.
\end{definition}
Note that the alternating form is multilinear, thus a tensor.
Also, it is totally antisymmetric: interchanging any two vectors changes the sign.
Equivalently,
$$[\underline{v_{\sigma(1)}},\underline{v_{\sigma(2)}},\ldots,\underline{v_{\sigma(n)}}]=\epsilon(\sigma)[\underline{v_1},\underline{v_2},\ldots,\underline{v_n}]$$
Moreover, $[\underline{e_1},\underline{e_2},\ldots,\underline{e_n}]=1$.\\
One can see immediately that
\begin{proposition}
    If the function $f:(F^n)^n\to F$ where $F=\mathbb R$ or $\mathbb C$ is multilinear, totally antisymmetric and $f(\underline{e_1},\underline{e_2},\ldots,\underline{e_n})=1$, then $f$ is uniquely determined.
\end{proposition}
\begin{proof}
    Trivial.
\end{proof}
\begin{proposition}
    If some vector(s) is $\underline{v_1},\underline{v_2},\ldots,\underline{v_n}$ is a linear combination of others, then $[\underline{v_1},\underline{v_2},\ldots,\underline{v_n}]=0$.
\end{proposition}
\begin{example}
    In $\mathbb C^4$, let $\underline{v_1}=(i,0,0,2),\underline{v_2}=(0,0,5i,0),\underline{v_3}=(3,2i,0,0),\underline{v_4}=(0,0,-i,1)$, then $[\underline{v_1},\underline{v_2},\underline{v_3},\underline{v_4}]=10i$
\end{example}
\begin{proposition}
    $[\underline{v_1},\underline{v_2},\ldots,\underline{v_n}]\neq 0$ if and only if $\{\underline{v_i}\}$ is an independent set.
\end{proposition}
\begin{proof}
    We have already shown the ``only if'' part, so it remains to show the other direction.
    If $\{\underline{v_i}\}$ is independent, then it constituted a basis, hence if $[\underline{v_1},\underline{v_2},\ldots,\underline{v_n}]=0$, then by multilinearity the alternating form will be zero everywhere, which is a contradiction.
\end{proof}
\begin{definition}[Definition of determinant]
    Consider $M\in M_{n\times n}(F)$ where $F=\mathbb R$ or $\mathbb C$ with columns $\underline{C_a}$, then the determinant of $M$ is
    \begin{align*}
        \det M&=[\underline{C_1},\underline{C_2},\ldots,\underline{C_n}]\\
        &=[M\underline{e_1},M\underline{e_2},\ldots,M\underline{e_n}]\\
        &=\epsilon_{ij\ldots kl}M_{i1}M_{j2}\cdots M_{k(n-1)}M_{ln}\\
        &=\sum_{\sigma\in S_n}\epsilon(\sigma)M_{\sigma(1)1}M_{\sigma(2)2}\cdots M_{\sigma(n)n}
    \end{align*}
\end{definition}
\begin{example}
    1. The definition of determinant here in general coincides with the cases in $2$ and $3$ dimensional cases.\\
    2. If $M$ is diagonal, then $\det M$ is the product of all diagonal entries.
    So $\det I=1$.\\
    3. If we have
    $$M=\left(\begin{array}{@{}ccc|c@{}}
        &&&0\\
        &A&&\vdots\\
        &&&0\\
        \hline
        0&\dots&0&1
    \end{array}\right)$$
    Then $\det M=\det A$
\end{example}
\subsection{Properties of Determinants}
If we multiply one of the columns (or rows) of the matrix $M$ by a scalar $\lambda$ to produce $M'$, we have $\det M'=\lambda\det M$.
Furthermore, if we interchange two adjascent columns, the determinant is negated.
We can see these directly from the definitive property of determinants.
In addition, $\det M\neq 0$ if and only if the columns are linearly independent.
\begin{proposition}
    For any $n\times n$ matrix $M$, $\det M=\det M^\top$.\\
    Equivalently, $[\underline{C_1},\underline{C_2},\ldots,\underline{C_n}]=[\underline{R_1},\underline{R_2},\ldots,\underline{R_n}]$.
\end{proposition}
\begin{proof}
    We have
    \begin{align*}
        [\underline{C_1},\underline{C_2},\ldots,\underline{C_n}]&=\sum_{\sigma\in S_n}\epsilon(\sigma)M_{\sigma(1)1}M_{\sigma(2)2}\cdots M_{\sigma(n)n}\\
        &=\sum_{\sigma\in S_n}\epsilon(\sigma)M_{1\sigma^{-1}(1)}M_{2\sigma^{-1}(2)}\cdots M_{n\sigma^{-1}(n)}\\
        &=\sum_{\sigma'\in S_n}\epsilon(\sigma')M_{1\sigma'(1)}M_{2\sigma'(2)}\cdots M_{n\sigma'(n)}\\
        &=[\underline{R_1},\underline{R_2},\ldots,\underline{R_n}]
    \end{align*}
    Since the map $\sigma\to\sigma'=\sigma^{-1}$ is an automorphism on $S_n$ and $\epsilon(\sigma)=\epsilon(\sigma')$.
\end{proof}
We can evaluate determinants by expanding rows or columns.
\begin{definition}
    For $M$ an $n\times n$ matrix,
    Define $M^{ia}$ be the determinant of the $(n-1)\times (n-1)$ matrix obtained by deleting the row $i$ and column $a$ of $M$.
    This is called a minor.
\end{definition}
\begin{proposition}\label{det_formula}
    $$\forall a,\det{M}=\sum_i(-1)^{i+a}M_{ia}M^{ia}$$
    $$\forall i,\det{M}=\sum_a(-1)^{i+a}M_{ia}M^{ia}$$
\end{proposition}
\begin{proof}
    Trivial but see later sections for the proof.
\end{proof}
By some trivial computation, we discover
\footnote{We definitely did not see that coming}
that matrices having many zeros would be easier to calculate, so it brings us to the ways to simplify the determinants.\\
The first thing we could do is row/column operations.
If we modify $M$ by mapping $\underline{C_i}\mapsto \underline{C_i}+\lambda\underline{C_j}, i\neq j$ (or equivalently on rows), then the determinant is not changed.
This follows immediate from multilinearity and total antisymmetry.\\
Plus, as we stated above, if we interchange $\underline{C_i},\underline{C_j},i\neq j$ (same for rows), then the determinant changes sign.
This can help us simplify the calculation greatly since we can produce a lot of $0$'s from there.
\begin{theorem}
    For any $n\times n$ matrices $M,N$, $\det(MN)=\det(M)\det(N)$.
\end{theorem}
\begin{lemma}
    $$\epsilon_{i_1i_2\ldots i_n}M_{i_1a_1}M_{i_2a_2}\cdots M_{i_na_n}=\epsilon_{a_1a_2\ldots a_n}\det{M}$$
\end{lemma}
\begin{proof}
    Trivial.
\end{proof}
\begin{proof}[Proof of the multiplicativity of determinants]
    By the preceding lemma,
    \begin{align*}
        \det(MN)&=\epsilon_{i_1i_2\ldots i_n}(MN)_{i_11}\cdots(MN)_{i_nn}\\
        &=\epsilon_{i_1i_2\ldots i_n}M_{i_1k_1}N_{k_11}\cdots M_{i_nk_n}N_{k_nn}\\
        &=\epsilon_{i_1i_2\ldots i_n}M_{i_1k_1}\cdots M_{i_nk_n}N_{k_11}\cdots N_{k_nn}\\
        &=\det{M}\epsilon_{k_1k_2\ldots k_n}N_{k_11}\cdots N_{k_nn}\\
        &=\det{M}\det{N}
    \end{align*}
    As desired.
\end{proof}
Note that this would mean that $\det$ is a group homomorphism.
There are a few consequences of the multiplicative property:
\begin{proposition}
    1. If $M$ is invertible, then $\det(M^{-1})=\det(M)^{-1}$.\\
    2. If $M$ is orthogonal, then $\det(M)=\pm 1$.\\
    3. If $M$ is unitary, then $|\det(M)|=1$.
\end{proposition}
\begin{proof}
    Trivial.
\end{proof}
\subsection{Minors, Cofactors and Inverses}
\begin{definition}
    Select a column $\underline{C_a}$ of matrix $M$, we can write $\underline{C_a}=M_{ia}\underline{e_i}$, so
    \begin{align*}
        \det M&=[\underline{C_1},\underline{C_2},\ldots,\underline{C_a},\ldots,\underline{C_n}]\\
        &=[\underline{C_1},\underline{C_2},\ldots,M_{ia}\underline{e_i},\ldots,\underline{C_n}]\\
        &=M_{ia}[\underline{C_1},\underline{C_2},\ldots,\underline{e_i},\ldots,\underline{C_n}]\\
        &=:\sum_iM_{ia}\Delta_{ia}
    \end{align*}
    So
    $$\Delta_{ia}=[\underline{C_1},\underline{C_2},\ldots,\underline{e_i},\ldots,\underline{C_n}]$$
    is called the cofactor.
    Note that the cofactor is exactly the determinant of the matrix removing the row $i$ and column $a$.
\end{definition}
\begin{proof}[Proof of Proposition \ref{det_formula}]
    We know, then, that
    $$\Delta_{ia}=(-1)^{i+a}M^{ia}$$
    by shuffling the rows and columns.
    Therefore we have
    $$\det M=\sum_iM_{ia}\Delta_{ia}=\sum_i(-1)^{i+a}M_{ia}M^{ia}$$
    since columns and rows do not have real difference (we can do a transpose anyways), the two statements are proved.
\end{proof}
Now we want to find our adjugate (or adjoint) $\tilde{M}$.
Note that in our above proof we can observe that $M_{ib}\Delta_{ia}=\det(M)\delta_{ab}$, so we can easily define $\tilde{M}_{ij}=\Delta_{ji}$.
So $\tilde{M}$ is the transpose of the matrix of cofactors, then the relation above becomes
$$M\tilde{M}=\det(M)I$$
as desired.
This justifies the existence of $\tilde{M}$ and $\det(M)$ in the full generality of $\mathbb C^N$ from beginning of this section.
\subsection{System of Linear Equations}
Consider a system of $n$ linear equations in $n$ unknowns $x_i$, written in vector-matrix form $A\underline{x}=\underline{b}$ where $A$ is some $n\times n$ matrix.
If $\det A$ is nonzero, then $A^{-1}$ exists, which implies an unique solution $\underline{x}=A^{-1}\underline{b}$.
But what if $\det A$ is zero?\\
We know that if $\underline{b}\notin\operatorname{Im}A$, then by definition there is no solution.
But if $\underline{b}\in\operatorname{Im}A$, then the entire (shifted) space $\underline{x_p}+\ker{A}$ where $A\underline{x_p}=\underline{b}$ is exactly all the solutions.
This can be seen from the linear superposition of the solutions.\\
Note that the formula $\underline{x_p}+\ker{A}$ also applied to the case for $\det A\neq 0$ since in that case $\ker{A}=\{\underline{0}\}$.
If $\underline{u_i}$ is a basis for $\ker{A}$, then the general solution is $\underline{x_p}+a_i\underline{u_i}$.
\begin{example}
    Consider $A\underline{x}=B$ where
    $$A=\begin{pmatrix}
        1&1&a\\
        a&1&1\\
        1&a&1
    \end{pmatrix},\underline{b}=\begin{pmatrix}
        1\\
        c\\
        1
    \end{pmatrix}$$
    Now $\det A=(a-1)^2(a+2)$.
    So if $a\notin \{1,-2\}$, then
    $$
    A^{-1}=\frac{1}{(a-1)(a+2)}
    \begin{pmatrix}
        -1&a+1&-1\\
        -1&-1&a+1\\
        a+1&-1&-1
    \end{pmatrix}$$
    So
    $$\underline{x}=A^{-1}\underline{b}=\frac{1}{(1-a)(a+2)}\begin{pmatrix}
        2-c-ca\\
        c-a\\
        c-a
    \end{pmatrix},c\in\mathbb R$$
    Geometrically, this solves to give a point.
    If $a=1$, then
    $$A=\begin{pmatrix}
        1&1&1\\
        1&1&1\\
        1&1&1
    \end{pmatrix}\implies\operatorname{Im}A=\left\{\lambda\begin{pmatrix}
        1\\
        1\\
        1
    \end{pmatrix}:\lambda\in\mathbb R\right\}$$
    So there is no solution if $c\neq 1$.
    For $c=1$, since $(1,0,0)^\top$ would be a particular solution, the solutions form the plane $(1,0,0)^\top+\ker A$, i.e. the general solution is of the form
    $$\begin{pmatrix}
        1-\lambda-\mu\\
        \lambda\\
        \mu
    \end{pmatrix},\lambda,\mu\in\mathbb R$$
    For $c=-2$, by again looking at the image we conclude that $c$ must be $-2$, in which case the same analysis gives us the general solution
    $$\begin{pmatrix}
        1+\lambda\\
        \lambda\\
        \lambda
    \end{pmatrix},\lambda\in\mathbb R$$
\end{example}
Let $\underline{R_1},\underline{R_2},\underline{R_3}$ be the rows of $A$, then
$$A\underline{u}=\underline{0}\iff \forall i\in\{1,2,3\},\underline{R_i}\cdot\underline{u}=0$$
Note that the latter system of equations described planes through the origin.
We know that the solution of the homogeneous problem is equivalent to finding the kernel of $A$.
If $\operatorname{rank}A=3$, then we must have $\underline{u}=\underline{0}$ since $\{\underline{R_i}\}$ would be independent.
If $\operatorname{rank}A=2$, then $\{\underline{R_i}\}$ spans a plane, so the kernel is living along the line of normal to the plane.
If $\operatorname{rank}A=1$, then the normals to the pairwise spanned planes are parallel, so the kernel is a plane.\\
Now consider instead $A\underline{u}=\underline{b}$, then it happens iff $\underline{R_i}\cdot \underline{u}=b_i$.
In this case, if $\operatorname{rank}A=3$, then the normals of the planes described by the the system intersects at a point, thus there is an unique solution.
If $\operatorname{rank}A=2$, then the planes may intersect, in which case the solution is a line, but it might not be the case.
If $\operatorname{rank}A=1$, then the planes may coincide, in which case they are all the same, thus we have a plane of solution.
But again this might not be the case.
Two of them may coincide but that is not enough.\\
But how do you solve the equations and find the kernels systematically?
\subsection{Gaussian Elimination and Echelon Form}
Consider $A\underline{x}=\underline{b}$ where $\underline{x}\in\mathbb R^n,\underline{b}\in\mathbb R^m$ and $A$ is an $m\times n$ matrix, then we can solve it by Gaussian Elimination.
In general, we can reorder rows by row operations to rewrite original system in simplier form.
Note that when we do row operations, we have to do it simultaneously on the matrix and on the vector.
Our aim is to finally transform the matrix to the following form
$$M=\begin{pmatrix}
    M_{11}&\star&\star&\dots&\star&\star&\dots&\star\\
    0&M_{22}&\star&\dots&\star&\star&\dots&\star\\
    0&0&M_{33}&\dots&\star&\star&\dots&\star\\
    \vdots&\vdots&\vdots&\ddots&\vdots&\vdots&\ddots&\vdots\\
    0&0&0&\dots&M_{kk}&\star&\dots&\star\\
    0&0&0&\dots&0&0&\dots&0\\
    \vdots&\vdots&\vdots&\ddots&\vdots&\vdots&\ddots&\vdots\\
    0&0&0&\dots&0&0&\dots&0
\end{pmatrix}$$
Note that the row rank equals the column rank.
We can use induction to prove that we can indeed use a way to obtain $M$ from $A$ in a finitely many number of row operations.
Note that $\det A=\pm\det M$