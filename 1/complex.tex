\section{The Field of Complex Numbers}
\subsection{Basic Definitions}
We construct the complex numbers in the following way:
\begin{definition}
    Consider the plane $\mathbb R^2$, we equip it with the \textit{complex multiplication} $\cdot:\mathbb R^2\times\mathbb R^2\to\mathbb R^2$ in a way that:
    $$(a,b)\cdot(c,d)=(ac-bd,ad+bc)$$
    If we denote $(x,y)$ by $x+iy$, the resulting field $(\mathbb R^2, +, \cdot, 1, 0)$ is called the complex numebrs, and is denotes by $\mathbb C$.
\end{definition}
\begin{proposition}
    $\mathbb C$ is indeed a field.
\end{proposition}
\begin{proof}
    Trivial.
\end{proof}
Note that $i^2=-1$
\begin{definition}
    The conjugate $\bar z$ or $z^*$ of $z=x+iy\in\mathbb C$ is defined as $x-iy$.
    The modulus $|z|$ of $z$ is defined as $\sqrt{z\bar z}=\sqrt{x^2+y^2}$.
    The argument $\arg z$ of $z$ is the angle $\theta$ such that $z=|z|(\cos\theta+i\sin\theta)$, taken mod $2\pi$.
\end{definition}
The last expression is called the polar form of a complex number.
It is obvious that $|zw|=|z||w|$ and that $|z|=|\bar z|$.
\begin{claim}
    The argument of any complex number $z$ is defined.
\end{claim}
\begin{proof}
    Trivial.
\end{proof}
It is worth to note that $\tan\theta = y/x$.
Although there are infinitely many angles that can make the equality in the polar form, we often take the principal value, i.e. within $(-\pi,\pi]$.
But it is the most convenient to take it as a value in $\mathbb R/2\pi\mathbb Z$.\\
Of course, the complex numbers inherits the geometrical meanings of the plane $\mathbb R^2$. The geometric representation of it is called the Argand diagram.
On the Argand diagram, the argument is essentially the (anticlockwise) angle between the vector representating the complex number and the positive real axis.
The modulus, at the same time, is the length of that vector.
The addition and substraction of the complex numbers are the same as what we did it with 2D vectors (i.e. parallelogram law).\\
There is an important theorem associated with the polynomial in the complex numbers.
\begin{theorem}[Fundamental Theorem of Algebra]
    Any nonconstant polynomial in $\mathbb C$ has a root.
\end{theorem}
One should note easily that it is equivalent to say that a nonconstant complex polynomial of order $n$ has exactly $n$ roots.
The theorem means that the process of field extensions ends at $\mathbb C$.
\begin{proof}
    Later.
\end{proof}
\begin{proposition}[Triangle Inequality]
    $|z+w|\le |z|+|w|$
\end{proposition}
\begin{proof}
    Since both sides are positive, it is equivalent to its squared form:
    $$(z+w)(\bar z+\bar w)\le z\bar z+w\bar w+2|z||w|\iff \frac{1}{2}(z\bar w+\bar zw)\le |z\bar w|$$
    But this is just to say that $\Re (z\bar w)\le |z\bar w|$, which is true.
\end{proof}
\begin{corollary}
    Replacing $w$ by $w-z$ gives $|w-z|\ge |w|-|z|$.
    By symmetry $|w-z|\ge |z|-|w|$, so we have the general form
    $$|w-z|\ge||z|-|w||$$
\end{corollary}
\begin{proposition}
    Let $z_1=r_1(\cos \theta_1+i\sin\theta_1)$ and $z_2=r_2(\cos\theta_2+i\sin\theta_2)$, then
    $$z_1z_2=r_1r_2(\cos(\theta_1+\theta_2)+i\sin(\theta_1+\theta_2))$$
    That is, $\arg z_1+\arg z_2=\arg z_1z_2\pmod{2\pi}$
\end{proposition}
\begin{proof}
    Just compound angle formula suffices.
    It is known to be the De Movrie's Theorem.
\end{proof}
\begin{corollary}
    $(\cos\theta+i\sin\theta)^n=\cos(n\theta)+i\sin(n\theta)$
\end{corollary}
\begin{proof}
    Induction shows the case $n\in\mathbb N$, for negative $n=-m$, we have
    $$\text{LHS}=(\cos(m\theta)+i\sin(m\theta))^{-1}=\cos(m\theta)-i\sin(m\theta)=\text{RHS}$$
    That establishes it
\end{proof}
\subsection{Exponential and Trigonometric Functions}
\begin{definition}
    $$e^z:=\sum_{n=0}^\infty \frac{z^n}{n!}$$
\end{definition}
The series converges for all $z$ since it absolutely converges.
Also due to absolute convergence, we can multiply and arrange the series, which gives
$$e^ze^w=e^{z+w}$$
Note as well that $e^0=1$ and $(e^z)^n=e^{nz}$ for $n\in\mathbb Z$.
\begin{definition}
    $\cos(z)=(e^{iz}+e^{-iz})/2$, which gives the series
    $$\sum_{n=0}^\infty(-1)^n\frac{z^{2n}}{(2n)!}$$
    Similarly $\sin(z)=(e^{iz}-e^{-iz})/(2i)$, so its series expansion is
    $$\sum_{n=0}^\infty(-1)^n\frac{z^{2n+1}}{(2n+1)!}$$
\end{definition}
By differentiating the series term by term,
$$(\sin z)^\prime=\cos z, (\cos z)^\prime =-\sin z, (e^z)^\prime=e^z$$
\begin{theorem}
    $$e^{iz}=\cos z+i\sin z$$
\end{theorem}
Note that $\cos z$ is not necessarily real, same for $\sin z$.
But if $z$ is real, they are.
\begin{lemma}
    $e^z=1\iff z=2in\pi$ for some $n\in\mathbb Z$.
\end{lemma}
\begin{proof}
    Write $z=x+iy$, then we have
    $$e^{x+iy}=e^xe^{iy}=e^x(\cos y+i\sin y),x,y\in\mathbb R$$
    So $e^x\cos y=1$ and $e^x\sin y=0$.
    Solving it gives $x=0, y=0\pmod{2\pi i}$
\end{proof}
We have the following general form of complex number
$$z=|z|e^{i\arg z}$$
\subsection{Roots of Unity}
\begin{definition}
    $z\in\mathbb C$ is called an $n^{th}$ root of unity if $z^n=1$
\end{definition}
To find all solutions to $z^n=1$, we write
$$z=re^{i\theta}$$
so $r^n=1$ and $iN\theta=2\pi in$ for some $n\in\mathbb Z$.
This gives $n$ distinct solutions:
$$z=e^{2\pi in/N}$$
where $n\in \{0,1,2,\ldots ,n-1\}$
The roots of unity lie on the unit circle on the Argand diagram.
They are the vertices of a regular $n$-gon.
\subsection{Logarithms and Complex Powers}
\begin{definition}
    Define $w=\log z$ by $e^w=z$ since we want $\log$ to be the inverse of $\exp$, which is not injective.
    So $\log$ is multi-valued.
    $$\log z=\log|z|+i\arg z\pmod{2\pi i}$$
    We can, of course, make it single-valued by restricting $\arg z$ to the principal branch $(-\pi,\pi]$ or $[0,2\pi)$.
    But in this case, we do not have $\log(ab)=\log a+\log b$
\end{definition}
In fact, one can prove that it is impossible to choose a $\log$ in the complex plane that lives up to every one of our expectations.
\begin{example}
    If $z=-1$, then $\log z=i\pi\pmod{2\pi i}$.
\end{example}
\begin{definition}
    We define $z^\alpha=e^{\alpha\log z}$ for any $\alpha, z\in\mathbb C$ where $z\neq 0$.
    Note that since $\log$ is multi-valued, the complex powers are multi-valued in general.
    They differ by a multiplicative factor in the form $e^{2n\pi i\alpha},n\in\mathbb Z$.
\end{definition}
If $\alpha\in\mathbb Z$, then the power is single-valued.
If $\alpha\in\mathbb Q$, it is finite-valued.
But in general, a complex power admits infinitely many values.
\begin{example}
    1. We want to calculate $i^i$. $\log i=\pi/2+2\pi ni$, so
    $$i^i=e^{i\log i}=e^{-\pi/2+2\pi n}$$
    where $n\in\mathbb Z$.\\
    2. We want to calculate $(1+i)^{1/2}$, so it equals
    $$e^{1/2\log\sqrt 2+i(\pi/4+2n\pi i)}=2^{1/4}e^{i\pi/8}/2$$
\end{example}
\subsection{Lines and Circles}
For a fixed $w\in\mathbb C$ such that $w\neq 0$, the set of points $z=\lambda w$ is a line through the origin in the direction of $w$.\\
By shifting $z=z_0+\lambda w$ is a line parallel to $z=\lambda w$ though $z_0$.\\
To write this in the form without the real parameter $\lambda$, we take the conjugate
$$\bar z=\bar z_0+\lambda\bar w$$
Conbining the two equations and eleminate $\lambda$, we have $\bar wz-w\bar z=\bar wz_0-w\bar z_0$.
This is one standard form of the equation of a line (there are others though).\\
The equation of a circle centered at $c\in\mathbb C$ with radius $\rho$ is given by $|z-c|=\rho$, which is to say
$$(z-c)(\bar z-\bar c)=\rho^2\iff z\bar z-c\bar z-\bar cz=\rho^2-c\bar c$$
The general form of a point on the circle is $c+\rho e^{i\theta}$ where $\theta$ is a real parameter.\\
Note that in the geometrical viewpoint of $\mathbb C$, $z\mapsto z+z_0$ is a translation, $z\mapsto\lambda z$ where $\lambda\in\mathbb R$ is a scaling, $z\mapsto ze^{i\theta}$ where $\theta\in\mathbb R$ is a rotation, $z\mapsto\bar z$ is a translation, $z\mapsto 1/z$ is an inversion.
Comparing with groups, translation, rotation, scaling, inversion generates the Mobius group which normally acts on $\mathbb C_\infty=\mathbb C\cup\{\infty\}\cong S^2$ by stereographic projection.
On $\mathbb C_\infty$, lines in $\mathbb C$ are circles.