\section{Eigenvalues and Eigenvectors}
For a linear map $T:V\to V$ where $V$ is a real or complex vector space, a vector $\underline{v}\neq\underline{0}$ is called an eigenvector of $T$ with eigenvalue $\lambda$ if $T\underline{v}=\lambda\underline{v}$.
Note that this happens if and only if
$$A\underline{v}=\lambda\underline{v}\iff (A-\lambda I)\underline{v}=\underline{0}$$
where $A$ is the matrix of $T$.
So $\lambda$ is an eigenvalue if and only if $\det(A-\lambda I)=0$ if and only if it is a root of the polynomial
$$\chi_A(t)=\det(A-tI)$$
which is a degree $n$ polynomial.
\begin{definition}
    For $A$ an $n\times n$ matrix, the characteristic polynomial $\chi_A(t)$ is defined as $\det(A-tI)$.
\end{definition}
This gives us a way to find the eigenvalues by looking at the characteristic polynomial and to find the eigenvectors by looking at the kernel of $A-\lambda I$.
\begin{example}
    1. Let $V=\mathbb C^3$ and
    $$A=\begin{pmatrix}
        2&i\\
        -i&2
    \end{pmatrix}$$
    So $\chi_A(t)=(2-t)^2-1$, so $\lambda$ is $1$ or $3$.
    For the eigenvalue $\lambda=1$, we find that the eigenvectors are spanned by $(1,i)$, and for $\lambda=3$, the eigenvectors are spanned by $(1,-i)$.
    \\
    2. Let $V=\mathbb R^2$, consider the shear
    $$A=
    \begin{pmatrix}
        1&1\\
        0&1
    \end{pmatrix}$$
    So $\chi_A(t)=(1-t)^2$, so $\lambda=1$.
    The eigenvectors are then spanned by $(1,0)$
    \\
    3. Let $V=\mathbb C^2$, and
    $$U=\begin{pmatrix}
        \cos\theta&-\sin\theta\\
        \sin\theta&\cos\theta
    \end{pmatrix}$$
    So $\chi_U(t)=t^2-2t\cos\theta+1$, so $\lambda=e^{\pm i\theta}$, so the eigenvectors are spanned by $(1,\pm i)$ respectively.\\
    4. Consider $V=\mathbb C^n$ and $A=\operatorname{diag}(\lambda_1,\lambda_2,\ldots,\lambda_n)$, then the eigenvalues are $\lambda_i$ and the corresponding eigenvectors are spanned by $\underline{e_i}$.
\end{example}
\begin{proposition}
    1. There exists at least one eigenvalue.
    In fact, there exists $n$ eigenvalues counting multiplicity.\\
    2. The trace of the matrix equals the sum of the eigenvalues (counting multiplicity as well).\\
    3. $\det A=\chi_A(0)$.\\
    4. If $A$ is real, then non-real eigenvalues occur in conjugates.
\end{proposition}
\begin{proof}
    Trivial.
\end{proof}
\begin{definition}
    For an eigenvalue $\lambda$ of a matrix $A$, define the eigenspace $E_\lambda$ is the set of all $\underline{v}$ with $A\underline{v}=\lambda\underline{v}$, so $E_\lambda=\ker(A-\lambda I)$.
    The geometrical multiplicity of $\lambda$ is $m_\lambda=\operatorname{null}E_\lambda$, and the algebraic multiplicity $M_\lambda$ is the multiplicity of $\lambda$ as a root of $\chi_A(t)$.
    So $\chi_A(t)=(t-\lambda)f(t)$ for $f(\lambda)\neq 0$.
\end{definition}
\begin{proposition}\label{alg_ge_geom}
    $M_\lambda\ge m_\lambda$.
\end{proposition}
\begin{proof}
    Postponed to later.
\end{proof}
Note that the strict inequality can happen in many non-trivial cases.
\begin{example}
    1. Take
    $$\begin{pmatrix}
        -2&2&-3\\
        2&1&-6\\
        -1&-2&0
    \end{pmatrix}$$
    Then $\chi_A(t)=(5-t)(t+3)^2$, so $5,-3$ are the eigenvalues.
    By calculation we find that $E_5=\operatorname{span}\{(1,2,-1)^\top\}$ and $E_{-3}=\operatorname{span}\{(-2,1,0)^\top,(3,0,1)^\top\}$, so $M_5=m_5=1,M_{-3}=m_{-3}=2$.\\
    2. Take
    $$A=\begin{pmatrix}
        -3&-1&1\\
        -1&-3&1\\
        -2&-2&0
    \end{pmatrix}$$
    Then $\chi_A(t)=-(t+2)^3$, so the only eigenvalue is $-2$, but $E_{-2}$ only has dimension $2$, therefore $M_{-2}=3\neq 2=m_{-2}$.\\
    3. Consider the reflection across some plane through the origin.
    It reflects every point on the normal without changing direction and fixes every point on the plane.
    So the only eigenvalues would be $\pm 1$ and $M_{-1}=m_{-1}=1,M_1=m_1=2$.\\
    4. Consider the generic rotation through some normal.
    So the only real eigenvalue would be $1$ and $M_1=m_1=1$ since the transformation changes every vector's direction except those on the normal, which it fixes.
\end{example}
\begin{proposition}
    Let $\underline{v_1},\underline{v_2},\ldots,\underline{v_r}$ be eigenvectors of a matrix $A$ with eigenvalues $\lambda_1,\lambda_2,\ldots,\lambda_r$.
    If the eigenvalues are distinct, then the eigenvectors are linearly independent.
\end{proposition}
\begin{proof}
    Suppose $\underline{w}=\sum_ia_i\underline{v_i}$, then $(A-\lambda I)\underline{w}=\sum_ia_i(\lambda_j-\lambda)\underline{v}$.
    Let $\underline{w}=\underline{0}$, and if we can find a linear combination where not all $a_i$'s is $0$, then we pick such a linear combination $\underline{0}=\sum_ia_i\underline{v_i}$ such that the number of nonzero $a_i$'s is the least.
    WLOG in this choice $a_1\neq 0$, then $\underline{0}=(A+\lambda I)\underline{0}=\sum_ia_i(\lambda_j-\lambda_1)\underline{v}$, which contradicts the minimality of the number of nonzero $a_i$ of our choice.
\end{proof}
\begin{proof}[Alternative proof]
    Given a linear relation $\underline{0}=\sum_ia_i\underline{v_i}$, then for any $j$, the consider
    $$\underline{0}=\prod_{i\neq j}(A-\lambda_iI)\sum_{i=1}^ra_i\underline{v_i}=a_j\underline{v_j}\prod_{i\neq j}(\lambda_j-\lambda_i)$$
    So $a_j=0$.
\end{proof}
\subsection{Diagonalizability}
\begin{proposition}
    For an $n\times n$ matrix $A$ acting on $V=\mathbb R^n$ or $\mathbb C^n$, the followings are equivalent:\\
    1. There exists a basis for $V$ consisting of eigenvectors of $A$.\\
    2. There exists an $n\times n$ invertible matrix $P$ such that $P^{-1}AP=D$ where is a diagonal matrix whose entries are the eigenvalues.
\end{proposition}
If either of these conditions hold, then we say $A$ is diagonalizable.
\begin{proof}
    For any matrix $P$, $AP$ has columns $A\underline{C_i}^{(P)}$, and $DP$ has columns
    $$\lambda_i\underline{C_i}^{(P)}$$
    so what it is saying is that every column of $P$ is an eigenvector of $A$.
    So if there is a basis of eigenvectors, we can just take $P$ to be the matrix whose columns are the basis eigenvectors.
    Conversely, given such a matrix $P$, its columns are a basis for $V$.
\end{proof}
\begin{example}
    1. Consider
    $$A=
    \begin{pmatrix}
        1&1\\
        0&1
    \end{pmatrix}$$
    So we only have one eigenvector, hence $A$ is not diagonalizable.\\
    2. Consider
    $$U=
    \begin{pmatrix}
        \cos\theta&-\sin\theta\\
        \sin\theta&\cos\theta
    \end{pmatrix}$$
    So it does not have real eigenvalues, but it has $2$ complex eigenvalues with linearly independent eigenvectors, thus it is diagonalizable over $\mathbb C^n$.
    Indeed, we can let
    $$P=
    \begin{pmatrix}
        1&1\\
        -i&i
    \end{pmatrix}$$
    Then
    $$P^{-1}UP=\begin{pmatrix}
        e^{i\theta}&0\\
        0&e^{-i\theta}
    \end{pmatrix}$$
\end{example}
\begin{proposition}
    If $A$ has $n$ distinct eigenvalues, then it is disgonalizable.
\end{proposition}
\begin{proof}
    Different eigenvalues give linearly independent eigenvectors.
\end{proof}
\begin{proposition}
    $A$ is diagonalizable if and only if $M_\lambda=m_\lambda$ for each eigenvalue $\lambda$ of $A$.
\end{proposition}
\begin{proof}
    The union of the basis vectors for the eigenspaces would be a basis for the entire space since we would have $\sum_{\lambda}M_\lambda=\sum_{\lambda}m_\lambda=n$.
\end{proof}
\begin{example}
    1. Again use
    $$A=\begin{pmatrix}
        -2&2&-3\\
        2&1&-6\\
        -1&-2&0
    \end{pmatrix}$$
    We already know from previous example that $\lambda=5,3$ and $M_5=m_5=1,M_{-3}=m_{-3}=2$, so $A$ is diagonalizable.
    Indeed, we have
    $$P^{-1}AP=\begin{pmatrix}
        5&&\\
        &-3&\\
        &&-3
    \end{pmatrix},P=\begin{pmatrix}
        1&-2&3\\
        2&1&0\\
        -1&0&1
    \end{pmatrix}$$
    2. Use the previous example
    $$A=\begin{pmatrix}
        -3&-1&1\\
        -1&-3&1\\
        -2&-2&0
    \end{pmatrix}$$
    Then $\lambda=-2$ and $M_{-2}=3>m_{-2}=2$, so $A$ is not diagonalizable.
    Indeed, if it is, then there is some invertible $P$ with $A=P(-2I)P^{-1}=-2I$ which is a contradiction.
\end{example}
\begin{definition}
    Two $n\times n$ matrices $A,B$ are similar if there is some invertible $n\times n$ matrix $P$ such that $A=PBP^{-1}$.
\end{definition}
This is trivially an equivalence relation.
\begin{proposition}
    If $A,B$ are similar, then $\operatorname{tr}B=\operatorname{tr}A$ and $\det B=\det A$.
    Also $\chi_B=\chi_A$.
\end{proposition}
In fact, if two matrices are similar, then we could get from one to the other by a change of basis.
The results here then are obvious.
\begin{proof}
    $$\operatorname{tr}(B)=\operatorname{tr}(P^{-1}AP)=\operatorname{tr}(APP^{-1})=\operatorname{tr}(A)$$
    $$\det(B)=\det(P^{-1}AP)=\det(P)^{-1}\det(A)\det(P)=\det(A)$$
    $$\chi_A(t)=\det(A-tI)=\det(P^{-1})\det(A-tI)\det(P)=\det(B-tI)=\chi_B(t)$$
    As desired.
\end{proof}
Note that if $A$ is duagonalizable, then so is $B$ since they are all in the same equivalence class containing some diagonal matrix.
\begin{proof}[Proof of Theorem \ref{alg_ge_geom}]
    Choose an eigenvalue $\lambda$ of an $n\times n$ matrix $A$ with $m_\lambda=r$ and a basis $\underline{v_1},\ldots,\underline{v_r}$ for $E_{\lambda}$.
    Extend this basis to the entire space by adding vectors $\underline{w_{r+1}},\ldots,\underline{w_n}$.
    Consider the matrix $P$ with columns $\underline{C_i}(P)=\underline{v_i}$ for $i=1,\ldots,r$ and $\underline{C_a}(P)=w_a$ for $a=r+1,\ldots,n$, then $AP=PB$ where $B$ is of the form
    $$B=\left(\begin{array}{@{}c|c@{}}
        \lambda I&*\\
        \hline
        0&\hat{B}
    \end{array}\right)$$
    Note that $P^{-1}AP=B$, so $\chi_A(t)=\chi_B(t)=(\lambda-t)^r\det(\hat{B}-tI)$, so $M_\lambda\ge m_\lambda$.
\end{proof}
\subsection{Diagonalisation of Hermitian and Symmetric Matrices}
Observe that if $A$ is Hermitian, then $(A\underline{v})^\dagger\underline{w}=\underline{v}^\dagger A\underline{w}$ for any complex vectors $\underline{v},\underline{w}$.
\begin{proposition}
    Hermitian matrices have real eigenvalues and orthogonal eigenvectors for distinct eigenvalues.
    In addition, if the matrix is actually real (hence symmetric), then for each eigenvalue stated above, we can choose real eigenvectors.
\end{proposition}
\begin{proof}
    Let $A$ be a Hemitian matrix and $\lambda$ an eigenvalue of it with eigenvector $\underline{v}$, then
    $$\bar{\lambda}\underline{v}^\dagger\underline{v}=(A\underline{v})^\dagger\underline{v}=\underline{v}^\dagger A\underline{v}=\lambda\underline{v}^\dagger\underline{v}$$
    So $\lambda=\bar\lambda$, thus $\lambda$ is real.\\
    If $\lambda,\mu$ are distinct eigenvalues with eigenvectors $\underline{v},\underline{w}$, then
    $$\mu\underline{v}^\dagger\underline{w}=\underline{v}^\dagger(A\underline{w})=(A\underline{v})^\dagger\underline{w}=\lambda\underline{v}^\dagger\underline{w}\implies (\lambda-\mu)\underline{v}^\dagger\underline{w}=0$$
    So $\underline{v}^\dagger\underline{w}=0$ since $\mu\neq\lambda$, thus $\underline{v}\perp\underline{w}$.\\
    For the last part, just choose the real or imaginary part (at least one of them is nonzero) of the eigenvectors.
\end{proof}
\begin{proposition}[Gram-Schmidt Orthogonalization]
    Given a linearly independent set of vectors $\underline{w_1},\underline{w_2},\ldots,\underline{w_k}$ in $\mathbb C^n$, we can construct a set $\underline{u_1},\underline{u_2},\ldots,\underline{u_k}$ of orthonormal vectors such that $\operatorname{span}(\{\underline{u_i}\})=\operatorname{span}(\{w_i\})$.
\end{proposition}
\begin{proof}
    We do it recursively by
    $$\underline{u_1}=\frac{1}{\|\underline{v_1}\|}\underline{v_1},\forall j>1,\underline{w_j}^{(1)}=\underline{w_j}-(\underline{u_1}^\dagger \underline{w_j})\underline{u_1}$$
    and
    $$\underline{u_r}=\frac{1}{\|\underline{w_r}^{(r-1)}\|}\underline{w_r}^{(r-1)},\forall j>r,\underline{w_j}^{(r)}=\underline{w_j}^{(r-1)}-(\underline{u_r}^\dagger \underline{w_j}^{(r-1)})\underline{u_r}$$
    Which, as one can easily verify, is valid.
\end{proof}
Using Gram-Schmidt, we can convert any basis for some eigenspace to an orthonormal basis $\mathscr B_\lambda$.
If $\lambda_i$ are distinct eigenvalues of some Hermitian basis, then by choosing this orthonormal basis for all eigenspaces, we have an orthonormal set of eigenvectors.
\begin{example}
    1. Take
    $$A=\begin{pmatrix}
        2&i\\
        -i&2
    \end{pmatrix},A^\dagger=A$$
    So the eigenvalues are $1,3$ with eigenvectors $2^{-1/2}(1,i)^\top,2^{-1/2}(1,-i)^\top$.
    Note that these two eigenvectors is an orthonormal basis for $\mathbb R^2$.
    So we can choose
    $$P=\frac{1}{\sqrt{2}}\begin{pmatrix}
        1&1\\
        i&-i
    \end{pmatrix}$$
    Which is unitary and $P^{-1}AP=\operatorname{diag}(1,3)$.\\
    2. Consider
    $$A=\begin{pmatrix}
        0&1&1\\
        1&0&1\\
        1&1&0
    \end{pmatrix}$$
    which is real and symmetric thus Hermitian and have eigenvalues $-1,2$ where the algebraic multiplicity of $-1$ is $2$.
    By calculation we find that the geometric multiplicity of $-1$ is also $2$.
    We can find an orthonormal basis for $E_{-1}$ by the Gram-Schmidt process.
    For example we can have 
    $$E_{-1}=\operatorname{span}\{2^{-1/2}(1,-1,0)^\top,6^{-1/2}(1,1,-2)^\top\}$$
    Also $E_{2}=\operatorname{span}\{3^{-1/2}(1,1,1)^\top\}$, thus we have found an orthonormal eigenbasis for $\mathbb R^3$.
    We can also find the corresponding (unitary hence orthogonal) $P$ which diagonalizes $A$.
\end{example}
\begin{theorem}
    Any $n\times n$ Hermitian matrix is diagonalizable by an unitary matrix.
    Equivalently, there is an orthonormal eigenbasis.
    In particular, for real and symmetric case, we can choose such that the orthonormal basis is real.
    \label{hermitian_diag}
\end{theorem}
\begin{proof}
    We have already established all of the theorem but the part of the matrix being diagonalizable, so it suffices to show that, which is sadly not examinable, but here goes:\\
    Consider a Hermitian $A=\mathbb C^n\to\mathbb C^n$.
    If $V$ is any subspace of $\mathbb C^n$ that is invariant under $A$, then for any $\underline{v_1},\underline{v_2}\in V$ we have $\underline{v_1}^\dagger (A\underline{v_2})=(A\underline{v_1})^\dagger\underline{v_2}$.
    We shall prove by induction on $m=\dim V\le n$ that such a subspace has an orthonormal basis of eigenvectors of $A$, which shall establish the result by taking $m=n$.
    As the initial step is trivial, we will show the induction step.\\
    Given $V$ as above, since we are working in $\mathbb C$, the characteristic equation always has a root, so there is some $\underline{v}\in V$ such that $A\underline{v}=\lambda\underline{v}$ for some $\lambda$.
    Let $W$ be the subspace $\{\underline{w}\in V:\underline{v}^\dagger\underline{w}=0\}$, then $\dim W=m-1<m$.
    Now
    $$\underline{w}\in W\implies \underline{v}^\dagger (A\underline{w})=(A\underline{v})^\dagger\underline{w}=\lambda\underline{v}^\dagger\underline{w}=0\implies A\underline{w}\in W$$
    So $W$ is invariant under $A$, so by induction hypothesis, there is an orthonormal basis for $W$ of eigenvectors of $A$, say $\{\underline{u_1},\ldots,\underline{u_{m-1}}\}$, then we deem $\underline{u_m}=|\underline{v}|^{-1}\underline{v}$, then $\{\underline{u_1},\ldots,\underline{u_m}\}$ gives an orthonormal basis of $V$, and the proof is done.
\end{proof}
\subsection{Quadratic Forms}
\begin{definition}
    A quadratic form is a function $\mathcal F:\mathbb R^n\to\mathbb R$ given by $\mathcal{F}(\underline{x})=\underline{x}^\top A\underline{x}$ where $A$ is a real symmetric $n\times n$ matrix.
    So $\mathcal F(\underline{x})=x_iA_{ij}x_j$.
\end{definition}
So by Theorem \ref{hermitian_diag}, we can diagonalize it by an orthogonal matrix $P$.
Then by letting $\underline{x'}=P^\top\underline{x}$, we have $\mathcal F(\underline{x})=\underline{x'}^\top D\underline{x'}$ where $D=\operatorname{diag}(\lambda_1,\lambda_2,\ldots,\lambda_n)$, so we sau that $\mathcal F$ is diagonalized.
Note that $\underline{x}=x_i'\underline{u_i}$ where $\underline{u_i}$ is the orthonormal eigenbasis.
Thus we can interpret what we have done as not changing the vector, but instead changing the basis in which we are working under.
This basis (and the corresponding coordinate axes) is called the principle axes of the quadratic form.
\begin{example}
    1. In $\mathbb R^2$, take
    $$A=\begin{pmatrix}
        \alpha&\beta\\
        \beta&\alpha
    \end{pmatrix}$$
    So we can diagonalize it by $\lambda_1=\alpha+\beta$, $\lambda_2=\alpha-\beta$ and $\underline{u_1}=2^{-1/2}(1,1)^\top,\underline{u_2}=2^{-1/2}(1,-1)^\top$
    Hence $\underline{x'}=(2^{-1/2}(x_1+x_2),2^{-1/2}(x_1-x_2))^\top$.
    For example, if we take $\alpha=3/2,\beta=-1/2$, then the quadratic form defines an ellipse.
    If we let $\beta=3/2,\alpha=-1/2$, then the quadratic from would be a hyperbola.\\
    2. In $\mathbb R^3$, by a suitable choice of basis, $\mathcal F(\underline{x})=\lambda_1x_1^2+\lambda_2x_2^2+\lambda_3^2$.
    If $\lambda_i>0$, the graph is an ellipsoid (a sphere stretched towards the directions of the axes).
    Otherwise, for example, the matrix that we have introduced in a former example 
    $$A=\begin{pmatrix}
        0&1&1\\
        1&0&1\\
        1&1&0
    \end{pmatrix}$$
    has eigenvalues $-1,2$ where the algebraic multiplicity of $-1$ is $2$.
    So on the principle axes $\mathcal F_A(\underline{x})=-x_1^2-x_2^2+2x_3^2$.
    Hence it gives a hyperboloid (obtained by rotating a hyperbola).
\end{example}
It has important application when we use the Hessians as the quadratic Form.
\subsection{Cayley-Hamilton Theorem}
\begin{theorem}[Cayley-Hamilton]
    $\chi_A(A)=0$.
\end{theorem}
\begin{remark}
    If $c_0I+c_1A+\cdots+c_nA^n$, then $A(c_1I+C_2A+\cdots+c_nA^{n-1})=-c_0I$.
    So if $c_0=\det A\neq 0$, we have
    $$A^{-1}=-\frac{1}{c_0}(c_1I+C_2A+\cdots+c_nA^{n-1})$$
\end{remark}
\begin{proof}
    The general $2\times 2$ case is immediate.
    The case for diagonalizable matrices is also obvious.
    The general case can be deal with a continuity argument or using modules.
    A proof by basic algebra is also presented in lecture but is deemed not elegant enough by the author, hence is omitted.
\end{proof}